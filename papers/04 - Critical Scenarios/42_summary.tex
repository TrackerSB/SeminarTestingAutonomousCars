\documentclass[oneside, notitlepage, twocolumn]{scrartcl}

\usepackage[utf8]{inputenc}
\usepackage[english]{babel}

\usepackage[acronym, automake, nopostdot, nomain, nonumberlist, numberedsection, section]{glossaries}
\usepackage{adjustbox}
\usepackage{biblatex}
\usepackage{booktabs}
\usepackage{geometry}
\usepackage{makecell}
\usepackage{tabularx}
\usepackage{titlesec}
\usepackage{url}
\usepackage{xcolor}
\usepackage{xspace}

\geometry{%
    left=1in,
    right=1in,
}

\newcommand{\tableheadline}[1]{\textbf{#1}}
\newcommand{\draft}[1]{\textcolor{red}{\textit{#1}}}
\newcommand{\eg}{e.\,g.\xspace}
\newcommand{\ie}{i.\,e.\xspace}

\renewcommand\cellalign{lt}

\setlength{\parskip}{1mm}
\setlength{\parindent}{0pt}
\titlespacing\section{0pt}{12pt plus 4pt minus 2pt}{2pt plus 2pt minus 2pt}

\title{\LARGE 4.2 --- Identification of critical cases of \glstext{adas} safety by \glstext{fot} based parameterization of a catalogue}
\subtitle{Summary}
\author{Stefan Huber}

\addbibresource{42references.bib}

\makeglossaries%
\loadglsentries{../acronyms.tex}

\begin{document}

\maketitle

\section{Summary}
Testing \gls{adas} and \gls{adf} is very time consuming that simulations are used instead.
The current approaches lack in their ability to generate (critical) test cases that are realistic and relevant especially if they are solely based on simulations.
This paper proposes a method that obtains realistic parameterization of predefined test scenarios of a catalogue which is based on an accident database.
The approach is demonstrated on the lane change scenario.\par
The first step is to collect experimental data by equipping a car with radar sensors and stereo cameras and driving with it on the highway repeatedly doing certain maneuvers.\par
The collected data is translated into a deterministic model.
The paper uses a piecewise function as well as a hyperbolic tangent function for describing a model.
Using this model a parameterization of the lane change scenario can be derived with sufficiently low parameter complexity.\par
When comparing the piecewise function with the hyperbolic tangent function the paper shows that the hyperbolic tangent function achieves higher approximation precision.
According to the paper the approximation errors can be further reduced using higher dimensions which result in combinatorial explosion.\par
As a result the proposed method concentrates on low parameter complexity and develops a criteria for classifying potential critical test cases.

\section{Critical Content}
For determining realistic parameterizations lots of data of real scenarios have to be collected.
So the effort is still very high even if using simulations.\par
The introduction mentions that the proposed method tries to solve problems of irrelevant or unrealistic or both scenarios but the paper presents no metric for determining whether a scenario really is relevant and realistic.
According to the introduction the method also deals with the problem of very large testing amount but at the same time it includes collecting  realistic data manually which is very time consuming.\par
I would not use the approach since the effort of collecting realistic data is very high.\par
I like the paper because the approach is well described and analysed including a discussion about trade-off between dimensionality and precision and a comparison of two methods to model the experimental data.

\section{Critical Questions}
\begin{enumerate}
    \item The experimental data is collected with a single car and probably a single driver.
        How realistic is the collected data concerning that the procedure of scenarios like lane change heavily depend on the physics of the car and the behavior of the driver (some may drive faster or more aggressive than others)?
    \item The paper characterizes in section III critical scenarios as scenarios exceeding an error threshold and occurring rarely.
        How realistic can the parameterization of critical test scenarios be if the process of collecting data does not focus explicitly on producing critical scenarios or even building accidents?
\end{enumerate}

\section{References}
\begingroup
\renewcommand{\section}[2]{}%
\nocite{*}
\printbibliography%
\endgroup

\end{document}
