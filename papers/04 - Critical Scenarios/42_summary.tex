\documentclass[oneside, notitlepage, twocolumn]{scrartcl}

\usepackage[utf8]{inputenc}
\usepackage[english]{babel}

\usepackage[acronym, automake, nopostdot, nomain, nonumberlist, numberedsection, section]{glossaries}
\usepackage{adjustbox}
\usepackage{biblatex}
\usepackage{booktabs}
\usepackage{geometry}
\usepackage{makecell}
\usepackage{tabularx}
\usepackage{titlesec}
\usepackage{url}
\usepackage{xcolor}
\usepackage{xspace}

\geometry{%
    left=1in,
    right=1in,
}

\newcommand{\tableheadline}[1]{\textbf{#1}}
\newcommand{\draft}[1]{\textcolor{red}{\textit{#1}}}
\newcommand{\eg}{e.\,g.\xspace}
\newcommand{\ie}{i.\,e.\xspace}

\renewcommand\cellalign{lt}

\setlength{\parskip}{1mm}
\setlength{\parindent}{0pt}
\titlespacing\section{0pt}{12pt plus 4pt minus 2pt}{2pt plus 2pt minus 2pt}

\title{\LARGE 4.2 --- Identification of critical cases of \glstext{adas} safety by \glstext{fot} based parameterization of a catalogue}
\subtitle{Summary}
\author{Stefan Huber}

\addbibresource{42references.bib}

\makeglossaries%
\loadglsentries{../acronyms.tex}

\begin{document}

\maketitle

\section{Summary}
Testing \glspl{adas} and \glspl{adf} is very time consuming and so simulations are used instead of real world tests.
The current approaches lack in their ability to generate (critical) test cases that are realistic and relevant especially if they are solely based on simulations.
This paper proposes a method that obtains realistic parameterization for predefined test scenarios of a catalogue which is based on an accident database.
The approach is demonstrated on the lane change scenario.\par
The first step is to collect experimental data.
In this case the authors used a car equipped with radar sensors and stereo cameras, drove on the highway and did repeatedly certain maneuvers.\par
The collected data is translated into a deterministic model.
The paper describes a piecewise function as well as a hyperbolic tangent function.
When comparing these functions the paper shows that the hyperbolic tangent function achieves higher approximation precision.
According to the paper the approximation errors can be further reduced using higher dimensions which may result in a combinatorial explosion.\par
Using the model a parameterization of the lane change scenario can be derived with sufficiently low parameter complexity.\par
As a result the proposed method is able to concentrate on low parameter complexity and develops a criteria for distinguishing potential critical test cases from other test cases.

\section{Critical Content}
For determining realistic parameterizations lots of data of real maneuvers has to be collected.
So the effort is still very high even if using simulations.\par
The introduction mentions that the proposed method tries to solve problems of irrelevant or unrealistic or both scenarios but the paper presents no metric for determining whether a scenario really is relevant and realistic.
According to the introduction the method also deals with the problem of very large testing amount but at the same time it includes collecting experimental data manually which is very time consuming too.\par
I would not use the approach since the effort of collecting realistic data is very high.\par
I do not like the paper because the real benefits of the approach are not summarized in the conclusions section and the goals described within the introduction do not precisely fit to what the method tries to achieve.

\section{Critical Questions}
\begin{enumerate}
    \item The experimental data is collected with a single car and probably a single driver.
        How realistic is the collected data concerning that the procedure of scenarios like lane change heavily depend on the physics of the car and the behavior of the driver (some may drive faster or more aggressive than others)?
    \item The paper characterizes in section III potential critical scenarios as scenarios exceeding an error threshold and occurring rarely.
        How realistic can the parameterization of critical test scenarios be if the process of collecting data does not focus explicitly on producing rare events, critical situations or even building accidents?
\end{enumerate}

\section{References}
\begingroup
\renewcommand{\section}[2]{}%
\nocite{*}
\printbibliography%
\endgroup

\end{document}
