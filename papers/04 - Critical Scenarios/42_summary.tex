\documentclass[oneside, notitlepage, twocolumn]{scrartcl}

\usepackage[utf8]{inputenc}
\usepackage[english]{babel}

\usepackage[acronym, automake, nopostdot, nomain, nonumberlist, numberedsection, section]{glossaries}
\usepackage{adjustbox}
\usepackage{biblatex}
\usepackage{booktabs}
\usepackage{geometry}
\usepackage{makecell}
\usepackage{tabularx}
\usepackage{titlesec}
\usepackage{url}
\usepackage{xcolor}
\usepackage{xspace}

\geometry{%
    left=1in,
    right=1in,
}

\newcommand{\tableheadline}[1]{\textbf{#1}}
\newcommand{\draft}[1]{\textcolor{red}{\textit{#1}}}
\newcommand{\eg}{e.\,g.\xspace}
\newcommand{\ie}{i.\,e.\xspace}

\renewcommand\cellalign{lt}

\setlength{\parskip}{1mm}
\setlength{\parindent}{0pt}
\titlespacing\section{0pt}{12pt plus 4pt minus 2pt}{2pt plus 2pt minus 2pt}

\title{\LARGE 4.2 --- Identification of critical cases of \glstext{adas} safety by \glstext{fot} based parameterization of a catalogue}
\subtitle{Summary}
\author{Stefan Huber}

\addbibresource{42references.bib}

\makeglossaries%
\loadglsentries{../acronyms.tex}

\begin{document}

\maketitle

\section{Summary}
Testing \gls{adas} and \gls{adf} is very time consuming that simulations are used instead.
The current approaches lack in their ability to generate (critical) test cases that are realistic and relevant especially if they are solely based on simulations.
This paper proposes a method that obtains realistic parameterization of predefined test scenarios of a catalogue which is based on an accident database.
The approach is demonstrated on the lane change scenario.\par
The first step is to collect experimental data by equipping a car with radar sensors and stereo cameras and driving with it on the highway repeatedly doing certain maneuvers.\par
The collected data is translated into a deterministic model.
The paper uses a piecewise function as well as a hyperbolic tangent function for describing a model.
Using this model a parameterization of the lane change scenario can be derived with sufficiently low parameter complexity.\par
When comparing the piecewise function with the hyperbolic tangent function the paper shows that the hyperbolic tangent function achieves higher approximation precision.
According to the paper the approximation errors can be further reduced using higher dimensions which result in combinatorial explosion.\par
\draft{conclusions\ldots}
\draft{%
    \begin{itemize}
        \item Criteria for identifying potential critical scenarios based on naturalistic data
        \item Two groups: helps building suitable parameterizations; need to be tested manually
    \end{itemize}
}

\section{Critical Content}
\draft{%
    \begin{itemize}
        \item Lots of testing for base line (167 cut-in maneuvers and 1700km highway driving)
        \item (typo: an characteristic of lane change)
        \item Realistic parameterization needs real data?!
    \end{itemize}
}

\section{Critical Questions}
% At least 2
\begin{enumerate}
    \item \draft{How realistic is it?}
    \item Second question
\end{enumerate}

\section{References}
\begingroup
\renewcommand{\section}[2]{}%
\nocite{*}
\printbibliography%
\endgroup

\end{document}
