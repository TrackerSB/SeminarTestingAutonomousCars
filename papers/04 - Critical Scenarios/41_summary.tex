\documentclass[oneside, notitlepage, twocolumn]{scrartcl}

\usepackage[utf8]{inputenc}
\usepackage[english]{babel}

\usepackage[acronym, automake, nopostdot, nomain, nonumberlist, numberedsection, section]{glossaries}
\usepackage{adjustbox}
\usepackage{biblatex}
\usepackage{booktabs}
\usepackage{geometry}
\usepackage{makecell}
\usepackage{tabularx}
\usepackage{titlesec}
\usepackage{url}
\usepackage{xcolor}
\usepackage{xspace}

\geometry{%
    left=1in,
    right=1in,
}

\newcommand{\tableheadline}[1]{\textbf{#1}}
\newcommand{\draft}[1]{\textcolor{red}{\textit{#1}}}
\newcommand{\eg}{e.\,g.\xspace}

\renewcommand\cellalign{lt}

\setlength{\parskip}{1mm}
\setlength{\parindent}{0pt}
\titlespacing\section{0pt}{12pt plus 4pt minus 2pt}{2pt plus 2pt minus 2pt}

\title{\LARGE 4.1 --- Automatic Generation of Safety-Critical Test Scenarios for Collision Avoidance of Road Vehicles}
\subtitle{Summary}
\author{Stefan Huber}

\addbibresource{41references.bib}

\makeglossaries%
\loadglsentries{../acronyms.tex}

\begin{document}

\maketitle

\section{Summary}
Since testing autonomous cars physically is very expensive and time consuming simulation-based testing is done to test autonomous cars.
Since the amount of possible test scenarios is huge and most of them are not challenging to the system this paper presents an constructive approach which modifies an initially uncritical test case in such way that its criticality increases.\\
To quantify the criticality the paper computes the scenarios drivable area which represents all positions the car under test can reach at certain time steps in a predefined time interval.
The solution space of a test case is defined as the subspace of the drivable area which maintains \draft{a safe motion through the scenario}.
The smaller the solution space is the more critical it is classified in the paper.\\
The reduction of the solution space is a quadratic optimization problem which is solved using \gls{ecos}.\\
As a result this approach is able to quantify and to increase the criticality of a given test case.

\section{Critical Content}
The approach assumes invariance in respect to position and orientation of traffic participants and the ego car.
So it is restricted to two dimensional test scenarios.
To extend the approach at this point is ``subject to future work''.\\
\draft{%
    All area which is occupied at some point in the test duration is not part of the drivable area.
}\\
\draft{The car knows the whole scenario instead of only seeing what a real cars sees.}\\
I like the paper because it describes precisely the mathematical environment including the variables, the definitions and formal concept of the optimizations.
It also sketches the used algorithms and mentions all software components as well as the source of the scenarios used.\\
Since the approach is entirely explained, all used programs are listed and it is constructively creating test scenarios I would use it for testing.\\
One problem arising when trying to implement the approach is that \gls{spot} which is used during the paper \draft{for dealing with trajectories of pedestrians which change during the simulation time} depends on the MatLab plugins ``Mapping Toolbox'' and ``Robotics System Toolbox'' which are expensive.
\draft{The latter is not even mentioned in the documentation of \gls{spot}.}

\section{Critical Questions}
\draft{%
    \begin{itemize}
        \item Three dimensional test cases?
        \item Safe areas around participants? Maybe based on their speed?
    \end{itemize}
}
\begin{enumerate}
    \item How fast does it converge against a local optimum?
    \item The paper states that the approach does not find the most critical situation.
        Is there a most critical test scenario and if there is how could the approach be extended to find this global minimum?
    \item The drivable area is shrunken by excluding the area any participant occupies at any point while the test is simulated.
        Can there be significant benefits when freeing space occupied by a participant when it is leaving the space?
\end{enumerate}

\section{References}
\begingroup
\renewcommand{\section}[2]{}%
\nocite{*}
\printbibliography%
\endgroup

\end{document}
