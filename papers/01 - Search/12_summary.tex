\documentclass[oneside, notitlepage, twocolumn]{scrartcl}

\usepackage[utf8]{inputenc}
\usepackage[english]{babel}

\usepackage[printonlyused]{acronym}
\usepackage{booktabs}
\usepackage{makecell}
\usepackage{tabularx}
\usepackage{url}
\usepackage{xcolor}

\newcommand{\tableheadline}[1]{\textbf{#1}}
\newcommand{\draft}[1]{\textcolor{red}{\textit{#1}}}
\newcommand{\eg}{e.\,g. }

\renewcommand\cellalign{lt}

\title{\LARGE Automatic Track Generation for High-End Racing Games Using Evolutionary Computation}
\subtitle{Summary}
\author{Stefan Huber}

\begin{document}

\maketitle

\section{Summary}
The paper discusses track generation for racing games which are not bound to a predefined reality like typical F1 championship games.
The main focus of the generation is on diversity of the track in sense of its curvature profile and its speed profile.
Therefore the approach uses a fitness function for each of these two aspects.
To maximize the fitness single-objective as well as multi-objective genetic algorithms are used.\\
For descriptions of tracks the paper explains two types of representations.
The first is the track representation in TORCS called phenotype.
It consists of an ordered list of segments where each segment is a straight having a length or a turn having a direction (left/right), an arc, a start radius and an end radius.
The second is the track encoding called genotype.
It describes the track as a list of control points connected with Bezier curves.
The control points are polar coordinates and consist of a radial coordinate, an angular coordinate and the slope of a track tangent.\\
\draft{The curvature and speed profiles of TORCS tracks are evaluated.}\\
\draft{To be continued\ldots}

\section{Critical Content}
The related work section mentions tons of other papers, games, works and used methods but it rarely states whether these approaches influences their approach and what way.
\draft{To be continued\ldots}
I like the idea of the approach of the paper but assuming that an equal distribution in the curvature profile and especially in the speed profile is too easy.
So a further parameter classifying a track as ``fast'', ``curvy'' or ``balanced'' should be introduced.

\twocolumn[{%
\section{Related Work}
% At least 4 not explicitly mentioned other papers
% How did I find it?
% Why did I choose it?
\begin{tabularx}{\textwidth}{XXXX}
    \tableheadline{Topic} & \tableheadline{Source} & \tableheadline{Search strategy} & \tableheadline{Why chosen?}\\
    \midrule
    Name 1 & Source 1 & Strategy 1 & Reasons 1\\
    \midrule
    Name 1 & Source 1 & Strategy 1 & Reasons 1\\
    \midrule
    Name 1 & Source 1 & Strategy 1 & Reasons 1\\
    \midrule
    Name 1 & Source 1 & Strategy 1 & Reasons 1\\
\end{tabularx}

\section{Critical Questions}
% At least 2
}]

\end{document}
