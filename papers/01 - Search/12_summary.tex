\documentclass[oneside, notitlepage, twocolumn]{scrartcl}

\usepackage[utf8]{inputenc}
\usepackage[english]{babel}

\usepackage[printonlyused]{acronym}
\usepackage{booktabs}
\usepackage{makecell}
\usepackage{tabularx}
\usepackage{url}
\usepackage{xcolor}

\newcommand{\tableheadline}[1]{\textbf{#1}}
\newcommand{\draft}[1]{\textcolor{red}{\textit{#1}}}
\newcommand{\eg}{e.\,g.}

\renewcommand\cellalign{lt}

\title{\LARGE Automatic Track Generation for High-End Racing Games Using Evolutionary Computation}
\subtitle{Summary}
\author{Stefan Huber}

\begin{document}

\maketitle

\section{Summary}
The paper discusses track generation for racing games which are not bound to a predefined reality.
The two main properties of generated tracks the approach focuses are its curvature profile and its speed profile defining a fitness function each.
To maximize the fitness the approach uses single-objective as well as multi-objective genetic algorithms.

\section{Critical Content}
The related work section mentions tons of other papers, games, works and used methods but it rarely states whether these approaches influences their approach and what way.

\twocolumn[{%
\section{Related Work}
% At least 4 not explicitly mentioned other papers
% How did I find it?
% Why did I choose it?
\begin{tabularx}{\textwidth}{XXXX}
    \tableheadline{Topic} & \tableheadline{Source} & \tableheadline{Search strategy} & \tableheadline{Why chosen?}\\
    \midrule
    Name 1 & Source 1 & Strategy 1 & Reasons 1\\
    \midrule
    Name 1 & Source 1 & Strategy 1 & Reasons 1\\
    \midrule
    Name 1 & Source 1 & Strategy 1 & Reasons 1\\
    \midrule
    Name 1 & Source 1 & Strategy 1 & Reasons 1\\
\end{tabularx}

\section{Critical Questions}
% At least 2
}]

\end{document}
