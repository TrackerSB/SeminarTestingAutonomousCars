\documentclass[oneside, notitlepage, twocolumn]{scrartcl}

\usepackage[utf8]{inputenc}
\usepackage[english]{babel}

\usepackage[printonlyused]{acronym}
\usepackage{booktabs}
\usepackage{makecell}
\usepackage{tabularx}
\usepackage{url}
\usepackage{xcolor}

\newcommand{\tableheadline}[1]{\textbf{#1}}
\newcommand{\draft}[1]{\textcolor{red}{\textit{#1}}}
\newcommand{\eg}{e.\,g.}

\renewcommand\cellalign{lt}

\title{\LARGE 1.1 - Employing Multi-Objective Search to Enhance Reactive Test Case Generation and Prioritization for Testing Industrial Cyber-Physical Systems}
\subtitle{Summary}
\author{Stefan Huber}

\begin{document}

\maketitle

\section{Summary}
The paper concentrates on \ac{CPS}s and simulation-based testing.
Since simulation-based testing is very complex and computationally expensive the paper introduces a search-based approach with focus on cost-effectiveness of generation of test cases and prioritization of the generated test cases.
To define cost-effectiveness the search considers multiple objectives including \ac{TET}, \ac{RC}, \ac{TCS} and prioritization-aware \ac{TCS} described by a fitness function.
The definition of prioritization of generated test cases focuses on the objectives \ac{TET} and \ac{FDC}.\\
Each of the initial test suites consists of randomly generated reactive test cases.
The approach generates new test suites  by applying a crossover operator and two mutation operators which add, remove, modify or exchange test cases within a test suite or interchange parts of test suites with other ones.
After each generation of a new test suite its cost-effectiveness is measured to assure its quality.\\
As a result choosing \ac{NSGA-II} as search algorithm (Apart from the defined similarity measures where \ac{MOEA/D} performed best), \(2/N\) as mutation rate where \(N\) is defined in section V.B.2 and 0.2 as crossover rate achieve the best performance for the approach.

\section{Critical Content}
The paper mentions crossover and mutation operators for creating test suites out of existing test suites.
The test cases on which the operators are applied are chosen randomly and the quality of the resulting test suites is checked afterwards.
Other strategies for selecting certain test cases beforehand are not discussed.\\
Section V.B.2 describes many parameters of configuration which are mostly justified in section V.E.
An actual research on the behavior and the influences on the results when changing these parameters is not done.\\
Based on \cite{bringmann13} the hyper volumne quality indicator should be used with care since it does not achieve a good multiplicative approximation ratio.\\
In general I like the paper especially because it is founded very and every step and choice is justified.

\twocolumn[{%
\section{Related Work}
\begin{tabularx}{\textwidth}{XXXX}
    \tableheadline{Topic} & \tableheadline{Source} & \tableheadline{Search strategy} & \tableheadline{Why chosen?}\\
    \midrule
    A fast and elitist multi objective genetic algorithm: NSGA-II & \url{https://ieeexplore.ieee.org/document/996017} & \makecell{Search engine\\(Search string:\\``nsga ii'')} & Since this paper introduces \ac{NSGA-II} it may help to understand why exactly this algorithm performed well.\\
    \midrule
    MOEA/D: A Multiobjective Evolutionary Algorithm Based on Decomposition & \url{https://dces.essex.ac.uk/staff/zhang/papers/moead.pdf} & \makecell{Search engine\\(Search string:\\``moea d'')} & This paper can be useful to understand why \ac{MOEA/D} outperforms \ac{NSGA-II} concerning similarity measures.\\
    \midrule
    Improving the optimization performance of NSGA-II algorithm by experiment design methods & \url{https://ieeexplore.ieee.org/document/6269589} & \makecell{IEEE xplore\\(Search string:\\``nsga ii'')} & Since \ac{NSGA-II} turned out to fit the best it may be worth to use an optimized version of NSGA-II.\\
    \midrule
    Adaptive multi-objective archive-based hybrid scatter search for segmentation in lung computed tomography imaging & \url{http://adsabs.harvard.edu/abs/2012EnOp...44..327B} & \makecell{Search engine:\\(Search string:\\``archive based\\hybrid scatter\\search'')} & This paper may be interesting for evaluation of an archive-based hybrid scatter search as they plan to.\\
\end{tabularx}

\section{Critical Questions}
\begin{enumerate}
    \item Is there an actual need finding strategies for choosing positions where to apply crossover or mutation operators?
    \item Do the results fundamentally change when certain fixed parameters/configurations are changed? Would it have made a difference if certain configurations would not be bound to specific values?
\end{enumerate}

\section{List of Abbreviations}
\begin{acronym}
    \acro{CPS}{Cyber-Physical System}
    \acro{FDC}{fault detection capability}
    \acro{MOEA/D}{Multi Objective Evolutionary Algorithm Based on Decomposition}
    \acro{NSGA-II}{Non-dominated Sorting Genetic Algorithm II}
    \acro{RC}{requirements coverage}
    \acro{TCS}{test case similarity}
    \acro{TET}{test execution time}
\end{acronym}
}]

\begin{thebibliography}{1}
    \bibitem{bringmann13}
        Karl Bringmann, Tobias Friedrich, Approximation quality of the hypervolume indicator, \url{https://www.researchgate.net/publication/256660343_Approximation_quality_of_the_hypervolume_indicator}, 2013
\end{thebibliography}
\end{document}
