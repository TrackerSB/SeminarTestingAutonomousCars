\documentclass[oneside, notitlepage, twocolumn]{scrartcl}

\usepackage[utf8]{inputenc}
\usepackage[english]{babel}

\usepackage[printonlyused]{acronym}
\usepackage{booktabs}
\usepackage{tabularx}
\usepackage{xcolor}

\newcommand{\tableheadline}[1]{\textbf{#1}}
\newcommand{\draft}[1]{\textcolor{red}{\textit{#1}}}

\title{1.1 - Employing Multi-Objective Search to Enhance Reactive Test Case Generation and Priorization for Testing Industrial Cyber-Physical Systems}
\subtitle{Summary}
\author{Stefan Huber}

\begin{document}

\maketitle

\section{Summary}
The paper concentrates on \ac{CPS}s and simulation-based testing which is currently the most commonly used technique for testing them.
Since simulation-based testing is very complex and computationally expensive the paper introduces a search-based approach with focus on cost-effectiveness of generation of test cases and priorization of the generated test cases.
To define cost-effectiveness the search considers multiple objectives including \ac{TET}, \ac{RC}, \ac{TCS} and priorization-aware \ac{TCS} described by a fitness function.
The definition of priorization of generated test cases focuses on the objectives \ac{TET} and \ac{FDC}.\\
%The paper describes theses objectives as follows:\\
%The \ac{RC} is the number of requirements covered by a test suite divided by the total number of requirements of a \ac{CPS}.
%[The other objects]
Each of the initial test suites consists of randomly generated reactive test cases.
The approach generates new test suites  by applying a crossover operator and two mutation operators which add, remove, modify or exchange test cases within a test suite or interchange parts of test suites with other ones.
After each generation of a new test suite its cost-effectiveness is measured to assure its quality.

\section{Critical Content}
% Where is the paper not accurate?
% Problems not tackled?
\draft{Crossover operator as well as the mutation operators are selecting a random place for splitting, combining and exchanging test suites and test cases.}\\

\twocolumn[{%
\section{Related Work}
% At least 4 not explicitely mentioned other papers
% How did I find it?
% Why did I choose it?
\begin{tabularx}{\textwidth}{XXXX}
    \tableheadline{Topic} & \tableheadline{Source} & \tableheadline{Search strategy} & \tableheadline{Why chosen?}\\
    \midrule
    Name 1 & Source 1 & Strategy 1 & Reasons 1\\
    \midrule
    Name 1 & Source 1 & Strategy 1 & Reasons 1\\
    \midrule
    Name 1 & Source 1 & Strategy 1 & Reasons 1\\
    \midrule
    Name 1 & Source 1 & Strategy 1 & Reasons 1\\
\end{tabularx}

\section{Critical Questions}
% At least 2

\section{List of Abbreviations}
\begin{acronym}
    \acro{CPS}{Cyber-Physical System}
    \acro{FDC}{fault detection capability}
    \acro{RC}{requirements coverage}
    \acro{TCS}{test case similarity}
    \acro{TET}{test execution time}
\end{acronym}
}]

\end{document}
