\documentclass[oneside, notitlepage, twocolumn]{scrartcl}

\usepackage[utf8]{inputenc}
\usepackage[english]{babel}

\usepackage[acronym, automake, nopostdot, nomain, nonumberlist, numberedsection, section]{glossaries}
\usepackage{adjustbox}
\usepackage{biblatex}
\usepackage{booktabs}
\usepackage{geometry}
\usepackage{makecell}
\usepackage{tabularx}
\usepackage{titlesec}
\usepackage{url}
\usepackage{xcolor}
\usepackage{xspace}

\geometry{%
    left=1in,
    right=1in,
}

\newcommand{\tableheadline}[1]{\textbf{#1}}
\newcommand{\draft}[1]{\textcolor{red}{\textit{#1}}}
\newcommand{\eg}{e.\,g.\xspace}

\renewcommand\cellalign{lt}

\setlength{\parskip}{1mm}
\setlength{\parindent}{0pt}
\titlespacing\section{0pt}{12pt plus 4pt minus 2pt}{2pt plus 2pt minus 2pt}

\title{\LARGE 3.2 --- Safety Verification of \glsname{adas} by Collision-Free Boundary Searching of a Parameterized Catalog}
\subtitle{Summary}
\author{Stefan Huber}

\addbibresource{32_references.bib}

\makeglossaries%
\loadglsentries{../acronyms.tex}

\begin{document}

\maketitle

\section{Summary}
To evaluate and verify \glspl{adas} and \glspl{adf} is even in the case of simulations very time consuming.
Therefore a test case catalog based on a traffic accident database was introduced in a previous paper to reduce the number of test scenarios to test.
This approach searches for a boundary dividing parameterizations for these scenarios into safe and unsafe conditions (\eg{} in terms of collision rate) using a \gls{gpc} based iterative algorithm.\par
The first iteration of \gls{gpc} starts with defining initial points which are classified as safe or unsafe.
As long as a given exit condition is not satisfied the boundary for all current points is generated and new input data is chosen which is also classified.
At the end the class boundary is numerically computed.
\draft{%
    \begin{itemize}
        \item Estimation of probability of these conditions
        \item \gls{gpc} based Input Design
        \item Probability shifts to either p=1 or p=0
    \end{itemize}
}

\section{Critical Content}
\draft{continue\ldots}\par
The approximation quality of the approach is not clear.
The quality is demonstrated on one example where ``it is desirable to find an exit condition depending on the current quality of the approximation''.
The authors use the predictive probability as exit condition which is according to the paper not applicable in practice due to numerical errors and discretization of the input which is supposed to mis-trigger the exit condition.\par
I would not use this approach since the paper contains multiple mistakes in grammar, misses labels like in Fig.10, has very few references, the approach itself how to choose new input data is not explained and the approximation quality is according to the paper unknown (see section III.C).

\section{Critical Questions}
\begin{enumerate}
    \item As the paper points out the approximation is sometimes getting worse temporarily presumably at edges of the true boundary.
        Is there another way to overcome this issue without defining a candidate set to pick new input data from?
    \item Is there any way to analyze the quality of the approximation?
\end{enumerate}

\section{References}
\begingroup
\renewcommand{\section}[2]{}%
\nocite{*}
\printbibliography%
\endgroup

\section{Related Work}
\draft{Apply approach to more complex ADAS in future work}
\begin{adjustbox}{angle=90}
\begin{tabularx}{\textwidth}{llX}
    \tableheadline{Ref.} & \tableheadline{Search strategy} & \tableheadline{Why chosen?}\\
    \midrule
    cite & \makecell{strategy\\(method)} & reasons\\
    \midrule
    cite & \makecell{strategy\\(method)} & reasons\\
    \midrule
    cite & \makecell{strategy\\(method)}& reasons\\
    \midrule
    cite & \makecell{strategy\\(method)}& reasons\\
\end{tabularx}
\end{adjustbox}

\end{document}
