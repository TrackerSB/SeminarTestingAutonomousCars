\documentclass[oneside, notitlepage, twocolumn]{scrartcl}

\usepackage[utf8]{inputenc}
\usepackage[english]{babel}

\usepackage[acronym, automake, nopostdot, nomain, nonumberlist, numberedsection, section]{glossaries}
\usepackage{adjustbox}
\usepackage{biblatex}
\usepackage{booktabs}
\usepackage{geometry}
\usepackage{makecell}
\usepackage{tabularx}
\usepackage{titlesec}
\usepackage{url}
\usepackage{xcolor}
\usepackage{xspace}

\geometry{%
    left=1in,
    right=1in,
}

\newcommand{\tableheadline}[1]{\textbf{#1}}
\newcommand{\draft}[1]{\textcolor{red}{\textit{#1}}}
\newcommand{\eg}{e.\,g.\xspace}
\newcommand{\ie}{i.\,e.\xspace}

\renewcommand\cellalign{lt}

\setlength{\parskip}{1mm}
\setlength{\parindent}{0pt}
\titlespacing\section{0pt}{12pt plus 4pt minus 2pt}{2pt plus 2pt minus 2pt}

\title{\LARGE 3.1 --- Delivering Test and Evaluation Tools for Autonomous Unmanned Vehicles to the Fleet}
\subtitle{Summary}
\author{Stefan Huber}

\addbibresource{31_references.bib}

\makeglossaries%
\loadglsentries{../acronyms.tex}

\begin{document}

\maketitle

\section{Summary}
Designing test cases for autonomous systems as well as live tests are very time consuming.
This paper introduces a method for generating test scenarios which are interesting in terms of being close to a performance boundary.
The performance boundary is defined as region of test cases where slight modifications to their configuration drastically change the resulting performance \eg{} are crucial whether a test succeeds or fails.\par
So in contrast to other approaches this approach deals with the relation between configuration and performance (in particular the performance boundary) instead of finding faulty behavior.\par
The approach starts with finding interesting/informative test scenarios using an iterative \gls{gpr} search algorithm which focuses on regions with high gradients since these indicate that there may be a performance boundary.
The \gls{gpr} provides a model of a mean value, the gradient of the mean and an estimated variance for each prediction point.\par
The resulting test scenarios are clustered based on the \gls{dbscan} which identifies the performance boundaries.

\section{Critical Content}
The definitions in the problem definition section are very brief may not be complete since the paper states that ``definitions include'' the listed ones.\par
Some explanations are given multiple times throughout the paper (\eg{} ``where small changes to the state space/scenario/configuration lead to large changes in performance'' or ``We use active learning strategy'').\par
I would not use the approach yet since the performance in terms of time is not given and the future work of the paper already includes scaling up the system for handling more dimensions and test cases.
So research on its computation time effort is coming.\par
I like the paper since the approach is not only limited to autonomous cars and it is able (according to the \gls{sut} section) to consider additional aspects like power consumption and remaining distance the \gls{sut} can reach.

\section{Critical Questions}
\begin{enumerate}
    \item The paper describes in the \gls{sut} section that the \gls{uuv} considers its power limitations.
        When the \gls{uuv} recognizes that it is not able not reach the goal position it changes the direction to reach a recovery point.
        Can a similar approach be used to consider fuel consumption of cars and to determine whether the car needs to drive to the next gas station?
    \item The boundary identification methods gradient and mean-shift outperform each other depending on whether the output space is continuous or discrete.
        Are there further aspects that influence the performance when comparing these methods?
\end{enumerate}

\section{References}
\begingroup
\renewcommand{\section}[2]{}%
\nocite{*}
\printbibliography%
\endgroup

\section{Related Work}
\begin{adjustbox}{angle=90}
\begin{tabularx}{\textwidth}{llX}
    \tableheadline{Ref.} & \tableheadline{Search strategy} & \tableheadline{Why chosen?}\\
    \midrule
    \cite{trucksFuel} & \makecell{IEEExplore search\\(search string:\\``fuel consumption'')} & The approach may be extended to consider fuel consumption of autonomous vehicles.\\
    \midrule
    \cite{higherDims} & \makecell{IEEExplore search\\(author search:\\``Galen E. Mullins'')}& The future work of the above paper includes increasing the considered dimensions in the configuration space.
        This paper discusses robot self-righting capabilities and introduces therefore algorithms able to deal with high-dimensional configuration spaces.\\
    \midrule
    \cite{accelerate} & \makecell{IEEExplore search\\(author search:\\``Galen E. Mullins'')} & The future work of the summarized paper mentions that the authors want to scale up the system.
    This paper focuses on accelerating the generation of test scenarios.\\
    \midrule
    \cite{gpr} & \makecell{IEEExplore search\\(search string:\\``gaussian process\\regression'')}& This paper proposes active learning strategies used with \gls{gpr} a similar way the above paper does.\\
\end{tabularx}
\end{adjustbox}

\end{document}
