\documentclass[oneside, notitlepage, twocolumn]{scrartcl}

\usepackage[utf8]{inputenc}
\usepackage[english]{babel}

\usepackage[acronym, automake, nopostdot, nomain, nonumberlist, numberedsection, section]{glossaries}
\usepackage{adjustbox}
\usepackage{biblatex}
\usepackage{booktabs}
\usepackage{geometry}
\usepackage{makecell}
\usepackage{tabularx}
\usepackage{titlesec}
\usepackage{url}
\usepackage{xcolor}
\usepackage{xspace}

\geometry{%
    left=1in,
    right=1in,
}

\newcommand{\tableheadline}[1]{\textbf{#1}}
\newcommand{\draft}[1]{\textcolor{red}{\textit{#1}}}
\newcommand{\eg}{e.\,g.\xspace}

\renewcommand\cellalign{lt}

\setlength{\parskip}{1mm}
\setlength{\parindent}{0pt}
\titlespacing\section{0pt}{12pt plus 4pt minus 2pt}{2pt plus 2pt minus 2pt}

\title{\LARGE 3.1 --- Delivering Test and Evaluation Tools for Autonomous Unmanned Vehicles to the Fleet}
\subtitle{Summary}
\author{Stefan Huber}

\addbibresource{31_references.bib}

\makeglossaries%
\loadglsentries{../acronyms.tex}

\begin{document}

\maketitle

\section{Summary}
\draft{%
    \begin{itemize}
        \item Designing test scenarios very expensive
        \item Live test also very time consuming
        \item Generate test scenarios informing on the expected performance
        \item Not focussing on fault detection but on relationship between scenario configuration and resulting performance
        \item Test scenario interesting/informative if close to a performance boundary
        \item Performance boundary = location in configuration space where discontinuity or large change in performance occurs
        \item (\eg{} slight change in position of obstacle causes to transition from success to failure)
        \item Uses \gls{gpr} to provide model of the mean value, the gradient of the mean and an estimated variance for each prediction point
        \item Iterative \gls{gpr} search algorithm
        \item Two steps: (1) search informative test cases, (2) find boundaries by clustering the test cases
    \end{itemize}
}

\section{Critical Content}
\draft{%
    \begin{itemize}
        \item Can not only be used for autonomous cars
        \item Some aspects explained multiple times (``where small changes to the state space lead to large changes in performance'', ``We follow an active learning strategy'')
    \end{itemize}
}

\section{Critical Questions}
% At least 2
\begin{enumerate}
    \item First question
    \item Second question
\end{enumerate}

\section{References}
\begingroup
\renewcommand{\section}[2]{}%
\nocite{*}
\printbibliography%
\endgroup

\section{Related Work}
% At least 4 not explicitly mentioned other papers
% How did I find it?
% Why did I choose it?
% Don´t use background papers.
% Don´t use surveys.
\draft{%
    \begin{itemize}
        \item Want to explore higher dimensions
        \item Want to scale up system to handle even larger numbers of dimensions and test cases
        \item Localized \gls{gpr}
        \item Non-stationary covariance functions for the \gls{gpr}
    \end{itemize}
}
\begin{adjustbox}{angle=90}
\begin{tabularx}{\textwidth}{llX}
    \tableheadline{Ref.} & \tableheadline{Search strategy} & \tableheadline{Why chosen?}\\
    \midrule
    cite & \makecell{strategy\\(method)} & reasons\\
    \midrule
    cite & \makecell{strategy\\(method)} & reasons\\
    \midrule
    cite & \makecell{strategy\\(method)}& reasons\\
    \midrule
    cite & \makecell{strategy\\(method)}& reasons\\
\end{tabularx}
\end{adjustbox}

\end{document}
