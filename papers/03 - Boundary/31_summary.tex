\documentclass[oneside, notitlepage, twocolumn]{scrartcl}

\usepackage[utf8]{inputenc}
\usepackage[english]{babel}

\usepackage[acronym, automake, nopostdot, nomain, nonumberlist, numberedsection, section]{glossaries}
\usepackage{adjustbox}
\usepackage{biblatex}
\usepackage{booktabs}
\usepackage{geometry}
\usepackage{makecell}
\usepackage{tabularx}
\usepackage{titlesec}
\usepackage{url}
\usepackage{xcolor}
\usepackage{xspace}

\geometry{%
    left=1in,
    right=1in,
}

\newcommand{\tableheadline}[1]{\textbf{#1}}
\newcommand{\draft}[1]{\textcolor{red}{\textit{#1}}}
\newcommand{\eg}{e.\,g.\xspace}
\newcommand{\ie}{i.\,e.\xspace}

\renewcommand\cellalign{lt}

\setlength{\parskip}{1mm}
\setlength{\parindent}{0pt}
\titlespacing\section{0pt}{12pt plus 4pt minus 2pt}{2pt plus 2pt minus 2pt}

\title{\LARGE 3.1 --- Delivering Test and Evaluation Tools for Autonomous Unmanned Vehicles to the Fleet}
\subtitle{Summary}
\author{Stefan Huber}

\addbibresource{31_references.bib}

\makeglossaries%
\loadglsentries{../acronyms.tex}

\begin{document}

\maketitle

\section{Summary}
Designing test cases for autonomous systems as well as live tests are very time consuming.
In contrast to other approaches this approach focuses on the relation between configuration and resulting performance instead of finding faulty behavior.\par
This paper introduces a method for generating test scenarios which are interesting in terms of test scenarios close to performance boundaries.
Therefore the performance boundary is defined as test cases where slight modifications to the configuration drastically change the resulting performance \eg{} are crucial to whether a tests succeeds or fails.\par
The approach starts with finding interesting/informative test scenarios using an iterative \gls{gpr} search algorithm which focuses on regions with high gradients since these may indicate that there is a performance boundary.
The \gls{gpr} provides a model of a mean value, the gradient of the mean and an estimated variance for each prediction point \draft{prediction point?}.\\
These test scenarios are clustered based on \gls{dbscan} which identifies the performance boundaries.

\section{Critical Content}
\draft{%
    \begin{itemize}
        \item Can not only be used for autonomous cars
        \item (Some aspects explained multiple times (``where small changes to the state space lead to large changes in performance'', ``We follow an active learning strategy''))
    \end{itemize}
}

\section{Critical Questions}
% At least 2
\begin{enumerate}
    \item First question
    \item Second question
\end{enumerate}

\section{References}
\begingroup
\renewcommand{\section}[2]{}%
\nocite{*}
\printbibliography%
\endgroup

\section{Related Work}
% At least 4 not explicitly mentioned other papers
% How did I find it?
% Why did I choose it?
% Don´t use background papers.
% Don´t use surveys.
\draft{%
    \begin{itemize}
        \item Want to explore higher dimensions
        \item Want to scale up system to handle even larger numbers of dimensions and test cases
        \item Localized \gls{gpr}
        \item Non-stationary covariance functions for the \gls{gpr}
    \end{itemize}
}
\begin{adjustbox}{angle=90}
\begin{tabularx}{\textwidth}{llX}
    \tableheadline{Ref.} & \tableheadline{Search strategy} & \tableheadline{Why chosen?}\\
    \midrule
    cite & \makecell{strategy\\(method)} & reasons\\
    \midrule
    cite & \makecell{strategy\\(method)} & reasons\\
    \midrule
    cite & \makecell{strategy\\(method)}& reasons\\
    \midrule
    cite & \makecell{strategy\\(method)}& reasons\\
\end{tabularx}
\end{adjustbox}

\end{document}
