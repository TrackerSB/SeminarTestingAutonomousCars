\documentclass[oneside, notitlepage, twocolumn]{scrartcl}

\usepackage[utf8]{inputenc}
\usepackage[english]{babel}

\usepackage[acronym, automake, nopostdot, nomain, nonumberlist, numberedsection, section]{glossaries}
\usepackage{adjustbox}
\usepackage{biblatex}
\usepackage{booktabs}
\usepackage{geometry}
\usepackage{makecell}
\usepackage{tabularx}
\usepackage{titlesec}
\usepackage{url}
\usepackage{xcolor}
\usepackage{xspace}

\geometry{%
    left=1in,
    right=1in,
}

\newcommand{\tableheadline}[1]{\textbf{#1}}
\newcommand{\draft}[1]{\textcolor{red}{\textit{#1}}}
\newcommand{\eg}{e.\,g.\xspace}
\newcommand{\ie}{i.\,e.\xspace}

\renewcommand\cellalign{lt}

\setlength{\parskip}{1mm}
\setlength{\parindent}{0pt}
\titlespacing\section{0pt}{12pt plus 4pt minus 2pt}{2pt plus 2pt minus 2pt}

\title{\LARGE 5.2 --- Simulation of Real Crashes as a Method for Estimating the Potential Benefits of Advanced Safety Technologies}
\subtitle{Summary}
\author{Stefan Huber}

\addbibresource{52references.bib}

\makeglossaries%
\loadglsentries{../acronyms.tex}

\begin{document}

\maketitle

\section{Summary}
Modern cars have an increasing number of secondary safety systems successfully integrated.
So to integrate primary safety functions successfully as well has growing importance.
This paper proposes a method for estimating the benefits of advanced safety systems by comparing it with real world crash scenarios using simulations.\par
The scenarios are taken from the \gls{gidas}.
The comparison is based on physical parameters resulting from the simulations of the ego vehicle in the real world scenario without and with a virtual prototype of a safety system.
The estimation of the benefits regards whether the accident could be avoided and/or mitigated, the collision speed and position of the participants.
Since the prototype is not restricted to include only one advanced safety system also interactions between multiple safety systems (\eg{} \gls{ebs} and \gls{ldw}), sensors, brakes or steering control and the benefits of combining them can be investigated.

\section{Critical Content}
The description of the approach is very precise and broad.
It explains every step taken in the simulations, every parameter considered at any step, problems occurring at certain points in the simulation like the visibility of other participants and also how they solved them (\eg{} detection lines).\par
In comparison, the explanation of calculating the benefits itself is rather short.
The method is able to state qualitatively whether there are benefits when using certain advanced safety systems but not quantitatively how much benefit there is (\eg{} collision avoidance criteria).\par
The estimation is heavily dependant on a detailed description of accident scenarios which may not be available for all scenarios/accidents of interest.\par
I would use the approach since many aspects are described in detail, solutions for problems arising during implementation/simulation are presented and the approach is illustrated and demonstrated very well considering multiple typical accident scenarios, different sensors and in which scenarios certain sensors have advantages.
Additionally combinations of advanced safety systems can be tested.\par
I really like the paper because of the previously mentioned precision and the number of illustrations.

\section{Critical Questions}
\begin{enumerate}
    \item The comparison of the simulation results is based on whether the accident could be avoided, the collision speed and position.
        Is there an explicit way to calculate benefits more quantitatively to be able to compare benefits of different safety systems?
    \item The paper states that the estimation depends on detailed descriptions of accident scenarios.
        Recognizing more details may result in increasing computation effort.
        Which extent of details or which concrete details are needed to have feasible computation effort but a minimum certainty about the estimation of benefits?
\end{enumerate}

\section{References}
\begingroup
\renewcommand{\section}[2]{}%
\nocite{*}
\printbibliography%
\endgroup

\end{document}
