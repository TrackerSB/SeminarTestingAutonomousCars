\documentclass[oneside, notitlepage, twocolumn]{scrartcl}

\usepackage[utf8]{inputenc}
\usepackage[english]{babel}

\usepackage[acronym, automake, nopostdot, nomain, nonumberlist, numberedsection, section]{glossaries}
\usepackage{adjustbox}
\usepackage{biblatex}
\usepackage{booktabs}
\usepackage{geometry}
\usepackage{makecell}
\usepackage{tabularx}
\usepackage{titlesec}
\usepackage{url}
\usepackage{xcolor}
\usepackage{xspace}

\geometry{%
    left=1in,
    right=1in,
}

\newcommand{\tableheadline}[1]{\textbf{#1}}
\newcommand{\draft}[1]{\textcolor{red}{\textit{#1}}}
\newcommand{\eg}{e.\,g.\xspace}
\newcommand{\ie}{i.\,e.\xspace}

\renewcommand\cellalign{lt}

\setlength{\parskip}{1mm}
\setlength{\parindent}{0pt}
\titlespacing\section{0pt}{12pt plus 4pt minus 2pt}{2pt plus 2pt minus 2pt}

\title{\LARGE 6.2 --- DeepRoad: \glstext{gan}-Based Metamorphic Testing and Input Validation Framework for Autonomous Driving Systems}
\subtitle{Summary}
\author{Stefan Huber}

\addbibresource{62references.bib}

\makeglossaries%
\loadglsentries{../acronyms.tex}

\begin{document}

\maketitle

\section{Summary}
Current testing techniques for automatically generating tests have problems including a lack of diversity in the generated tests and a low accuracy if the training and application domains mismatch.
This may compromise the efficacy and reliability of the testing methods.\par
To solve these problems this paper introduces an unsupervised \gls{dnn} based framework for automatically testing the consistency of \gls{dnn} based autonomous systems and online validation called DeepRoad.\par
At first DeepRoad synthesizes large amounts of diverse driving scenarios by applying \glspl{gan} and without using simple image transformations.
The main focus is on changing the weather conditions of the images.\par
Afterwards DeepRoad checks the consistency of the \gls{dnn} under test using these synthetic images.
A \gls{dnn} is considered consistent if the deviation between the steering predictions of the initial and the transformed image is small.\par
DeepRoad also validates input images by measuring the distance of the input image to the training images \draft{using their VGGNet features}.
According to the paper the input validation turns out to be not appropriate.\par
DeepRoad detected thousands of inconsistent behaviors among three well recognized \gls{dnn} based driving systems in Udacity and validated input images which potentially enhance the robustness of the system.

\section{Critical Content}
To clarify the goals of the paper starts with broadly explaining the proposed benefits and advantages of DeepRoad compared to other systems like DeepXplore and DeepTest.\par
The paper also describes that DeepRoad is in contrast to other systems able to allow online input validation which could be used to warn the driver if the autonomous is currently not able to handle the situation itself (\eg{} snow covered road).\par
The authors also mention the testing oracle problem which is essential for the consistency check.\par
\draft{continue\ldots}\par
I would use the approach since the results shown in figure 8 are very convincing about the capability of changing the weather conditions of an image.\par
I like the paper especially since it explicitly distinguishes this approach from other approaches and compares it to them.
\draft{%
    \begin{itemize}
        \item Contains section about threads to validity
        \item Input validation == Distance between input and training images \(=>\) new images accepted? \(=>\) distance not proper metric
        \item Only one with hard rain and one with heavy snow \(=>\) Same car, same camera, same angle, similar/same road (only highway?, accidents?)
    \end{itemize}
}

\section{Critical Questions}
\draft{%
    \begin{enumerate}
        \item Paper criticises other approaches lacking being able to handle disturbed input images. It does not evaluate whether DeepRoad successfully deals with that.
    \end{enumerate}
}

\section{References}
\begingroup
\renewcommand{\section}[2]{}%
\nocite{*}
\printbibliography%
\endgroup

\end{document}
