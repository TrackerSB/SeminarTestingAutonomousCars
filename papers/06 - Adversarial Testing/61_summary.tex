\documentclass[oneside, notitlepage, twocolumn]{scrartcl}

\usepackage[utf8]{inputenc}
\usepackage[english]{babel}

\usepackage[acronym, automake, nopostdot, nomain, nonumberlist, numberedsection, section]{glossaries}
\usepackage{adjustbox}
\usepackage{biblatex}
\usepackage{booktabs}
\usepackage{geometry}
\usepackage{makecell}
\usepackage{tabularx}
\usepackage{titlesec}
\usepackage{url}
\usepackage{xcolor}
\usepackage{xspace}

\geometry{%
    left=1in,
    right=1in,
}

\newcommand{\tableheadline}[1]{\textbf{#1}}
\newcommand{\draft}[1]{\textcolor{red}{\textit{#1}}}
\newcommand{\eg}{e.\,g.\xspace}
\newcommand{\ie}{i.\,e.\xspace}

\renewcommand\cellalign{lt}

\setlength{\parskip}{1mm}
\setlength{\parindent}{0pt}
\titlespacing\section{0pt}{12pt plus 4pt minus 2pt}{2pt plus 2pt minus 2pt}

\title{\LARGE 6.1 --- DeepTest: Automated Testing of Deep-Neural-Network-driven Autonomous Cars}
\subtitle{Summary}
\author{Stefan Huber}

\addbibresource{61references.bib}

\makeglossaries%
\loadglsentries{../acronyms.tex}

\begin{document}

\maketitle

\section{Summary}
In autonomous driving \glspl{dnn} have an increasing importance due to the advantages of the past years.
A major problem of \glspl{dnn} is that they are heavily dependent on training data which has to be collected and labeled manually or generated using unguided simulations.
Besides these strategies are very time consuming both have the problem of not finding corner cases or erroneous behavior and having a high diversity like different driving conditions.
This paper introduces a systematic testing tool for automatically detecting erroneous behaviors of \glspl{dnn} called DeepTest.
The paper explains its design and its implementation and evaluates the tool.\\
To find as many faulty behaviors of a \gls{dnn} as possible DeepTest tries to generate test cases which maximize the number of activated neurons of the \gls{dnn}.
The investigations of this approach shows that transformations shearing, scaling, translating, rotating them or applying different realistic driving conditions like rain, fog or blurring increase the number of different activated neurons.
The neuron coverage can be further increased according to the paper by combining such transformations to generate additional test cases.

\section{Critical Content}
The paper starts with explaining the spectacular/tremendous/great/significant progress in machine learning and \glspl{dnn} excessively.
It also describes broadly the progress certain companies made, the millions of miles they drove without any human interaction and that many US states already passed legislation to enable testing and deployment of autonomous vehicles.\\
According to the paper table 1 clearly shows that fatal crashes often happen under rare corner cases.
Since the table contains exactly three very short examples it is not convincing that fatal crashes are a result of unseen corner cases.\\
I would use the approach since it allows to test \glspl{dnn} but with a comparable low amount of data needed to be collected manually due to it generates much additional data using the described transformations.\\
I like the paper since it explains the main ideas, the reasoning behind the neuron coverage and the strategies for maximizing it very accurately and it has a section dedicated to threads to validity.
\draft{%
    \begin{itemize}
        \item (Introduction very broadly explains problems in trying to test \glspl{dnn} like traditional software)
        \item Is neuron coverage enough? Similar problems as with branch or line coverage? (Neuron coverage may not be useful if inputs are not likely to appear) \(=>\) generating realistic synthetic images
        \item (``We implement to the best of our knowledge.'' Really?!)
        \item (Found thousands of erroneous behavior; Udacity; test three top performing \gls{dnn}) Three times word by word
        \item (RQ3.i has slight problems with sentence structure)
    \end{itemize}
}

\section{Critical Questions}
\draft{%
    \begin{enumerate}
        \item Realistic synthetic inputs by transforming seed images \(=>\) Increase dependence on seed images
    \end{enumerate}
}

\section{References}
\begingroup
\renewcommand{\section}[2]{}%
\nocite{*}
\printbibliography%
\endgroup

\end{document}
