\documentclass[oneside, notitlepage, twocolumn]{scrartcl}

\usepackage[utf8]{inputenc}
\usepackage[english]{babel}

\usepackage[acronym, automake, nopostdot, nomain, nonumberlist, numberedsection, section]{glossaries}
\usepackage{biblatex}
\usepackage{booktabs}
\usepackage{makecell}
\usepackage{tabularx}
\usepackage{titlesec}
\usepackage{url}
\usepackage{xcolor}
\usepackage{xspace}

\newcommand{\tableheadline}[1]{\textbf{#1}}
\newcommand{\draft}[1]{\textcolor{red}{\textit{#1}}}
\newcommand{\eg}{e.\,g.\xspace}

\renewcommand\cellalign{lt}

\setlength{\parskip}{1mm}
\setlength{\parindent}{0pt}
\titlespacing\section{0pt}{12pt plus 4pt minus 2pt}{2pt plus 2pt minus 2pt}

\title{\LARGE 2.2 --- Testing of Autonomous Vehicles Using Surrogate Models and Stochastic Optimization}
\subtitle{Summary}
\author{Stefan Huber}

\addbibresource{references22.bib}

\makeglossaries%
\loadglsentries{../acronyms.tex}

\begin{document}

\maketitle

\section{Summary}
One of the main problems when testing autonomous cars is the huge input space when testing them.
This paper proposes a method for finding regions of faulty behavior with a given subspace \(\hat{P}\) of the overall parameter space \(P\) by using surrogate models and stochastic optimization.

\section{Critical Content}
% Where is the paper not accurate?
% Problems not tackled?
% Approach usable for testing?
% Did you like the paper?
\draft{%
    \begin{itemize}
        \item Outcome heavily depends on quality of the chosen values.
        \item The overall execution time of the proposed algorithm depends strong on the parameters.
        \item (page 5 right top) samples not saved between test runs even though in a real scenario beneficial.
        \item \(\eta\) not clearly defined.
    \end{itemize}
}

\section{Critical Questions}
% At least 2

\printglossary[type=\acronymtype, title=List of Abbreviations]

\section{References}
\begingroup
\renewcommand{\section}[2]{}%
\nocite{*}
\printbibliography%
\endgroup

\twocolumn[{%
\section{Related Work}
% At least 4 not explicitly mentioned other papers
% How did I find it?
% Why did I choose it?
% Don´t use background papers.
% Don´t use surveys.
\begin{tabularx}{\textwidth}{lXX}
    \tableheadline{Reference} & \tableheadline{Search strategy} & \tableheadline{Why chosen?}\\
    \midrule
    cite & \makecell{strategy\\(method)} & reasons\\
    \midrule
    cite & \makecell{strategy\\(method)} & reasons\\
    \midrule
    cite & \makecell{strategy\\(method)}& reasons\\
    \midrule
    cite & \makecell{strategy\\(method)}& reasons\\
\end{tabularx}
}]

\end{document}
