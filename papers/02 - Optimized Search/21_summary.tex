\documentclass[oneside, notitlepage, twocolumn]{scrartcl}

\usepackage[utf8]{inputenc}
\usepackage[english]{babel}

\usepackage[acronym, automake, nopostdot, nomain, numberedsection, section]{glossaries}
\usepackage{biblatex}
\usepackage{booktabs}
\usepackage{makecell}
\usepackage{tabularx}
\usepackage{titlesec}
\usepackage{url}
\usepackage{xcolor}

\newcommand{\tableheadline}[1]{\textbf{#1}}
\newcommand{\draft}[1]{\textcolor{red}{\textit{#1}}}
\newcommand{\eg}{e.\,g. }

\renewcommand\cellalign{lt}

\setlength{\parskip}{1mm}
\setlength{\parindent}{0pt}
\titlespacing\section{0pt}{12pt plus 4pt minus 2pt}{2pt plus 2pt minus 2pt}

\title{\LARGE Testing Vision-Based Control Systems Using Learnable Evolutionary Algorithms}
\subtitle{Summary}
\author{Stefan Huber}

\addbibresource{references21.bib}

\makeglossaries%
\newacronym{aeb}{AEB}{Automated Emergency Breaking}
\newacronym{adas}{ADAS}{Advanced Driving Assistance System}

\begin{document}

\maketitle

\section{Summary}
The paper discusses finding critical test cases for simulation-based testing of vision-based control systems like \glspl{adas} based on evolutionary search algorithms.
The approach is explained on their results on the \gls{aeb}.\par
To improve the effectiveness of the evolutionary multi-objective search algorithms the approach builds decision trees after each iteration.
These learn the characteristics of critical test scenarios and identify regions of critical test cases.
The next iterations concentrate on such regions to generate more critical test scenarios.\par
Every \gls{adas} is described as a tuple \((S, O, I, D, C)\) where \(S\) is the set of immobile objects, \(O\) is the set of mobile objects, \(I\) is the set of initial states of the mobile objects, \(D\) is the domain where the variables are in and \(C\) is a set of boolean constraints to \(S\) and \(I\).
% TODO Read starting at section 4.3

\section{Critical Content}
% Where is the paper not accurate?
% Problems not tackled?
\draft{%
    \begin{itemize}
        \item Assumption pedestrian linear trajectory
        \item They say they studied ten \glspl{adas} but only mention the \gls{aeb} (Section 4.1). Since cooperation with industry?
        \item They do not mention the mutation rate or whether they fixed it.
    \end{itemize}
}

\section{Critical Questions}
% At least 2

\printglossary[type=\acronymtype, title=List of Abbreviations]

\section{References}
\begingroup
\renewcommand{\section}[2]{}%
\nocite{*}
\printbibliography%
\endgroup

\twocolumn[{%
\section{Related Work}
% At least 4 not explicitly mentioned other papers
% How did I find it?
% Why did I choose it?
\begin{tabularx}{\textwidth}{lXX}
    \tableheadline{Reference} & \tableheadline{Search strategy} & \tableheadline{Why chosen?}\\
    \midrule
    cite & strategy & reasons\\
    \midrule
    cite & strategy & reasons\\
    \midrule
    cite & strategy & reasons\\
    \midrule
    cite & strategy & reasons\\
\end{tabularx}
}]

\end{document}
