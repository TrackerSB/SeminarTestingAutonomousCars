\documentclass{scrartcl}

\usepackage{color}

\title{Sketch of Final Presentation}
\subtitle{Demonstration}
\author{Stefan Huber}

\newcommand{\draft}[1]{\textcolor{red}{\textit{#1}}}

\setlength{\parskip}{1mm}
\setlength{\parindent}{0pt}

\begin{document}

\maketitle%

\section{Summary of main idea for demonstration}
The underlying idea is to show an initially uncritical scenario and how the approach modifies it step by step to increase its criticality.

\section{Rough structure of the presentation}
The final presentation is going to have two main sections.\par
In the first section I explain the goal of the approach including that testing of autonomous cars is done using simulations and therefore finding critical test scenarios is crucial.
To distinguish this approach from others I point out that this approach does not try to find critical scenarios within a subspace of the overall input space but to explicitly construct a critical test case by modification.
At this point I show the initial scenario which is going to be modified using the approach.\par
In the second part I start with defining criticality in terms of drivable area.
Therefore I present the fully extended drivable area of the initial scenario and explain that all possible positions and orientations are considered.
Then I show a sequence of images visualizing the development of the scenario and the drivable area resulting.

\end{document}

